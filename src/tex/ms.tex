% Define document class
\documentclass[]{aastex631}
\usepackage{showyourwork}
\usepackage{macros}
\usepackage{amsmath}
\usepackage{enumerate}
\usepackage{acro}
\acsetup{patch/longtable=false}


\DeclareAcronym{LVK}{short=LVK, long=LIGO Virgo KAGRA}
\DeclareAcronym{O5}{short=O5, long=the fifth LIGO Virgo KAGRA observing run}

\DeclareAcronym{GW}{short=GW, long=gravitational wave}
\DeclareAcronym{CBC}{short=CBC, long=compact binary coalescence}
\DeclareAcronym{BBH}{short=BBH, long=binary black hole}

\DeclareAcronym{GP}{short=GP, long=Gaussian process, long-plural=es}

% Begin!
\begin{document}

% Title
\title{No need to know: astrophysics-agnostic gravitational wave cosmology}
%\title{No need to know: astrophysics-agnostic spectral siren cosmology}
%\title{No need to know: spectral siren cosmology with an astrophysics-agnostic mass distribution of gravitational-wave sources}
%\title{No need to know: gravitational-wave cosmology for the astrophysically-ignorant}

% Author list
%\author{@afarah18}
\author[0000-0002-6121-0285]{Amanda M. Farah}
\email{afarah@uchicago.edu}
\affiliation{Department of Physics, University of Chicago, Chicago, IL 60637, USA}

\author{Thomas A. Callister}
\affiliation{Kavli Institute for Cosmological Physics, The University of Chicago, Chicago, IL 60637, USA}

\author[0000-0002-7213-3211]{Jose Mar\'ia Ezquiaga}
%\email{jose.ezquiaga@nbi.ku.dk}
\affiliation{Niels Bohr International Academy, Niels Bohr Institute, Blegdamsvej 17, DK-2100 Copenhagen, Denmark}

\author[0000-0002-0147-0835]{Michael Zevin}
\affiliation{The Adler Planetarium, 1300 South DuSable Lake Shore Drive, Chicago, 60605, IL, USA}

\author{Daniel E. Holz}
\affiliation{Department of Physics, University of Chicago, Chicago, IL 60637, USA}
\affiliation{Kavli Institute for Cosmological Physics and Enrico Fermi Institute, The University of Chicago, Chicago, IL 60637, USA}
\affiliation{Department of Astronomy \& Astrophysics, The University of Chicago, Chicago, IL 60637, USA}


% Abstract
\begin{abstract}
\Acp{GW} from merging compact objects are the only transient sources that encode direct information about their luminosity distances. 
Paired with their redshifts, a Hubble diagram can be constructed to probe the Universe's expansion.  
``Spectral sirens'' use features in the mass distribution of \ac{GW} sources to extract this redshift information without the need for accompanying electromagnetic observations.
So far, this technique has only been applied with simple parametric representations of the mass distribution. 
However, the use of an incorrect mass model inevitably leads to biases in the cosmological inference, an acute problem given our current lack of understanding of the true source population.
Furthermore, it is commonly presumed that the exact shape of the mass distribution must be known \emph{a priori} to obtain unbiased measurements of cosmological parameters with the spectral siren methodology.
We demonstrate that prior knowledge of the \ac{GW} source mass distribution is not necessary in order to accurately infer cosmological parameters with spectral sirens.
We do this by constructing a flexible, Gaussian process-based model for the \ac{GW} source population to analyze simulated \ac{GW} data consistent with expectations for the next \acl{LVK} observing run.
We find that both the source mass model and cosmological parameters are correctly reconstructed, and predict a $\variable{output/nonparh0percent.txt}\%$ measurement of \Ho{} and a $\variable{output/Hz_percent.txt}\%$  measurement of $H(z=\variable{output/mostsensitivez.txt})$ ($1\sigma$ uncertainties).
This astrophysics-agnostic spectral siren technique will be key to deliver precise and unbiased cosmological constraints with future observations and in the presence of redshift evolution of the mass spectrum.  
\end{abstract}

d\section{}
\textcolor{red}{Showyourwork Guidelines:}
\begin{itemize}
    \item Always edit \texttt{ms.tex} in Overleaf
    \item Always edit \texttt{bib.bib} in Overleaf
    \item Always edit \texttt{macros.sty} in Overleaf
    \item Always edit anything in the \texttt{figures} directory using git (with a python script that should generate the figures)
    \item Always edit anything in the \texttt{output} directory using git (with a python script that should generate the output)
\end{itemize}

\section{Introduction}
\label{sec:intro}
Like light, \acp{GW} are redshifted as they propagate across the universe, thereby becoming imprinted with the cosmic expansion history.
Unlike light, however, the form of \ac{GW} signals are known from first principles and their selection effects are extremely well understood.
This allows for a precise estimate of each \ac{GW} catalogs' completeness and an unbiased measurement of the true \ac{GW} population. 
Additionally, the \ac{GW} signals detected by the LIGO \citep{aasi_advanced_2015}, Virgo \citep{acernese_advanced_2014}, and KAGRA \citep{akutsu_overview_2021} detectors provide direct measurements of the distance to their sources.
This makes them ``standard sirens'': powerful probes of cosmological parameters that circumvent the cosmological distance ladder~\citep{schutz_determining_1986,holz_using_2005}. 
This is particularly useful in the context of the current ``Hubble tension:'' a disagreement between multiple methods of measuring the local expansion rate of the universe with electromagnetic observations \citep{freedman_measurements_2021}.

A well-known application of standard sirens was the multi-messenger event GW170817 \citep{abbott_multi-messenger_2017}, whose clear association with a host galaxy provided a precise redshift measurement, allowing for a direct measurement of the Hubble constant, \Ho{} \citep{ abbott_gravitational-wave_2017}.
External redshift information can also come from galaxy catalogs, which provide redshift distributions for each \ac{GW} signal, allowing for a probabilistic measurement of \Ho{} when multiple \ac{GW} detections are combined \citep[e.g.][]{del_pozzo_inference_2012, chen_two_2018, fishbach_standard_2019, gray_cosmological_2020, gwtc3_cosmo, gair_hitchhikers_2023}.

However, external information about \ac{GW} source redshifts need not be available in order to use them as standard sirens.
\ac{GW} signals provide direct measurements of each source's luminosity distance, $d_L$, and redshifted, or detector frame, masses, $m_{\det}= m_{\source}(1+z)$ \citep[e.g.][]{chen_mass-redshift_2019}.
Therefore, if the source frame mass is known, each event provides a direct mapping between luminosity distance and redshift, allowing for a measurement of the rate of expansion of the universe at the time the \ac{GW} signal was emitted, $H(z)$.
In practice, unless tidal information is available \citep{messenger_measuring_2012}, the source frame mass is unknown.
However, it is possible to know the location of features in the source frame mass distribution.
The whole mass distribution therefore acts similarly to an electromagnetic spectrum, where the location of spectral features provides a redshift measurement.
The method of using the mass distribution of \ac{GW} sources to measure cosmological parameters has therefore been coined ``spectral sirens'' \citep{ezquiaga_spectral_2022}.
Spectral sirens were first demonstrated to be a feasible method to measure the Hubble constant by \cite{chernoff_gravitational_1993,Taylor:2011fs} using the binary neutron star mass distribution, and extended to the \ac{BBH} mass distribution by \cite{farr_future_2019}. 
Spectral siren analyses have since been implemented by the LIGO--Virgo--KAGRA Collaborations using the latest \ac{GW} catalog \cite{gwtc3_cosmo}.

Perhaps because of the analogy to electromagnetic spectra in which the location of the source frame spectral features is well-known, or because of the examples given in \cite{farr_future_2019} and \cite{chernoff_gravitational_1993}, it is commonly presumed that the exact shape of the mass distribution must be known \emph{a priori} to do this measurement, from theoretical expectations of e.g. the location of a pulsational-pair instability pileup, upper mass gap, or maximum neutron star mass. 
This would certainly be problematic, as mismodeling of the specific feature would hinder the accuracy of the method and introduce biases in the cosmological inference \citep{Mukherjee:2021rtw,mastrogiovanni_importance_2021,pierra_study_2023}.
Such systematic bias is illustrated in Figure~\ref{fig:O5_GP}, where we show the result of fitting simulated data with a model that has the true mass distribution contained within its prior and one that does not. 
The latter does not accurately infer the true value of \Ho{}, whereas the former does.

This is particularly troubling in light of the fact that there remains much uncertainty over the cause, location, and robustness of currently-identified features in the \ac{GW} source mass distribution.
While several features have been identified in the current data \citep{tiwari_exploring_2022,edelman_aint_2022, abbott_population_2023,edelman_cover_2023}, including three overdensities, two underdensities, a maximum and minimum black hole mass, only three of these are statistically robust \citep{farah_things_2023}.
Additionally, the morphology of the mass distribution is a result of the interplay between several complex physical processes pertaining to massive stars, such as binary stellar evolution, pair instability, globular cluster dynamics, and stellar winds, each of which has large theoretical uncertainties \citep[e.g.][]{ zevin_constraining_2017, mapelli_binary_2020,marchant_evolution_2023}. 
These large theoretical uncertainties mean that accurate predictions for the shape of the \ac{GW} source mass distribution are a considerable challenge.
If such \emph{a priori} knowledge of the mass distribution's morphology were requisite for a spectral siren method, the effectiveness of such a measurement would be significantly hampered.

However, the information in the spectral siren measurement comes solely from the assumption that all \acp{CBC} follow the same mass distribution.\footnote{More precisely, the spectral siren method is also valid if \acp{CBC} have different mass distributions at different redshifts, so long as the evolution of the mass distribution does not exactly mimic cosmology, and the model for the mass distribution allows for evolution \citep{ezquiaga_spectral_2022}. }
Therefore the location, shape, and existence of all features in the mass distribution can be inferred simultaneously with $H(z)$. 

In this work, we explicitly demonstrate that no \emph{a priori} knowledge about the shape of the \ac{CBC} mass spectrum is necessary to use the spectral siren methodology. 
We do this by inferring $H(z)$ with a flexible, non-parametric model for the mass distribution of \acp{CBC} (blue lines in Figure~\ref{fig:O5_GP}). 
This model makes minimal prior assumptions about the shape of the mass distribution, enabling it to accurately infer a wide range of morphologies.
Despite its flexibility, it is able to consistently obtain unbiased measurements of cosmological parameters, showing that nonparametric methods are not only sufficient for a spectral siren measurement, they can also mitigate systematic effects in the measurement caused by model misspecification.

We find that using our non-parametric model model allows for a $\variable{output/nonparh0percent.txt}\%$ measurement of \Ho{} and a $\variable{output/Hz_percent.txt}\%$ measurement of $H(z=\variable{output/mostsensitivez.txt})$ during \ac{O5}, when the detectors will reach their design sensitivity. %, and a \result{X}\% measurement with only \result{X} weeks of observing with a next-generation detector network. 
While constraints on \Ho{} are used in this work to quantify the accuracy and precision of our non-parametric method, we emphasize that spectral sirens are able to constrain $H(z)$ out to redshifts that are inaccessible by electromagnetic observations, especially with next-generation detectors \citep{ezquiaga_spectral_2022, Chen:2024gdn, You:2020wju}\footnote{See example spectral siren cosmological inference using parametric mass spectrum models: \href{https://github.com/ezquiaga/spectral_sirens}{https://github.com/ezquiaga/spectral\_sirens}.}.
Applied to data from next-generation detectors, this method will therefore be able to constrain the dark energy equation of state parameter, $w$, at redshifts $\gtrsim 2$.

This paper is organized as follows: Section~\ref{sec:ss} explains how cosmological parameters are inferred from the mass distribution of \ac{GW} sources (i.e. the spectral siren method).
Section~\ref{sec:model} describes the non-parametric mass distribution we develop for use within the spectral siren method.
We present the results of using parametric and non-parametric mass distributions in Section~\ref{sec:results}, as well as projections for future constraints on $H(z)$.
In Section~\ref{sec:discussion} we discuss the implications of our results and outline future work.

The code used to generate all simulated data, perform the population inference, and create all figures in this paper is made publicly available at \GitHubURL{} under an MIT license, and was enabled by the \showyourwork package \citep{Luger2021}.

\begin{figure*}
    \centering
    \includegraphics[width=\textwidth]{figures/O5_pm.pdf}
    \caption{Spectral siren measurement for an \ac{O5}-like catalog with the correct parametric model (\plp, green), a deliberately incorrect parametric model (broken power law, orange) and the non-parametric model presented in this work (\acl{GP}, blue).
    The left panel shows the recovered source frame primary mass distribution for each model, and the simultaneously-inferred posteriors on \Ho{} are shown in the right panel.
    The mass distribution and \Ho{} value used to generate the data are shown by a solid black line in each panel. 
    The deliberately incorrect parametric model fails to recover the true mass distribution and therefore produces a biased estimate of \Ho{}, whereas both the correct and non-parametric models recover  the mass distribution and \Ho{}.
    As the true mass distribution is unknown for real observations, using a non-parametric model mitigates systematic uncertainty that would otherwise arise from mismodeling the \ac{CBC} population, though it does introduce additional statistical uncertainty.
    }
    \label{fig:O5_GP}
    \script{pm_H0_twopanel.py}
\end{figure*}

\section{Cosmology with an astrophysically-agnostic mass distribution model}
\label{sec:methods}
\subsection{The Spectral Siren Method}
\label{sec:ss}
To simultaneously infer cosmological parameters and the population of GW sources, we employ a hierarchical Bayesian analysis.
This allows us to ``undo'' the selection effects of \ac{GW} detectors to obtain a true, or astrophysical, population and constrain the cosmic expansion history.

Given \ac{GW} data $\{d\}$ that contains $N_{\text{evs}}$ detected \ac{GW} signals, each with parameters $\theta_i$, the posterior probability of the \ac{GW} sources being drawn from an underlying population and background cosmology described by hyper-parameters $\Lambda$ is \citep{loredo_handling_2009, Taylor:2011fs, mandel_extracting_2019}
\begin{equation}
    p(\Lambda|d) \propto p(\Lambda) e^{-N_{\text{exp}}(\Lambda)} \prod_i^{N_{\text{evs}}} \frac{\diff N}{\diff t_{\det} \diff \theta_i} (\theta_i;\Lambda) \, .
    \label{eq:inhomog-poisson}
\end{equation}
Here, $\Lambda$ is the set of population parameters, or hyper-parameters, and $p(\Lambda)$ is the prior on $\Lambda$.
$\frac{\diff N}{\diff t_{\det} \diff \theta} (\theta;\Lambda)$ is the detector frame merger rate density of \acp{BBH} conditioned on hyper-parameters $\Lambda$, this quantity is also referred to as the population model. 
Following \citet{callister_parameter-free_2023}, we use a semicolon to explicitly indicate that this is a function of $\Lambda$, not a density over $\Lambda$.
$N_{\text{exp}}(\Lambda)$ is the expected number of detections given $\Lambda$ and the \ac{GW} detector sensitivity, and is calculated using a Monte Carlo sum over $N_{\text{inj}}$ found signals injected into the data stream \citep[see][for a detailed explanation of this process]{essick_estimating_2021, essick_precision_2022}.
In this work, we have restricted our analysis to the \ac{BBH} mass distribution.
However, the method can be trivially extended to the full mass distribution of \acp{CBC} \citep[e.g.][]{fishbach_does_2020, ezquiaga_spectral_2022}.

Since the event parameters are not perfectly measured, we marginalize $\frac{\diff N}{\diff t_{\det} \diff \theta} (\theta;\Lambda)$ over the likelihood of each event's parameters given the data.
Practically, this is done by a Monte Carlo average over each event's posterior samples $\{\theta_j\}_i$, dividing out the prior used when inferring those posterior samples, $\pi_{PE}(\theta)$:
\begin{equation}
    \frac{\diff N}{\diff t_{\det} \diff \theta_i} (\theta_i;\Lambda) = \frac{1}{N_{\text{samples}}}\sum_j^{N_{\text{samples}}} \frac{\diff N}{\diff t_{\det} \diff \theta_{i,j}} (\theta_{i,j};\Lambda) \frac{1}{\pi_{PE}(\theta_{i,j})} .
    \label{eq:single-event-likelihood}
\end{equation}
Finally, $p(\Lambda)$ is the prior on the population and parameters on the background cosmology, which we discuss in further detail below.
We infer the posterior in Equation~\ref{eq:inhomog-poisson} using the no-u-turn sampler for Hamiltonian Monte Carlo within \texttt{numpyro} \citep{hoffman_no-u-turn_2011, bingham_pyro_2019,phan_composable_2019}. 

For our population model, we focus on the masses and redshifts, neglecting spins as they do not contain cosmological information.
For simplicity, we additionally assume that the population has equal masses, $m_1=m_2$.
Thus, $\theta = \{m_1,z\}$. 
The population can be modeled with different levels of complexity and flexibility. 
As described in detail in Section~\ref{sec:model}, we model the distribution of primary masses in an agnostic way, using a \acl{GP}. 
We also compare this non-parametric reconstruction with different parametric models.  
For all models considered in this work, we assume a uniform-in-comoving volume redshift distribution, as in Equation~\ref{eq:underlying redshift dist}.

The spectral siren method of measuring cosmological parameters relies on the assumption that either all compact binaries follow the same mass distribution at all redshifts, or that any redshift evolution of the mass distribution does  not exactly mimic cosmology (i.e. all features cannot exhibit a monotonic rightward shift with redshift) \citep{ezquiaga_spectral_2022}. For the redshift range considered in this work, we assume the mass distribution has no redshift evolution.
The likelihood in Equation~\ref{eq:inhomog-poisson} is therefore maximized when the relationship between redshift and luminosity distance makes all source frame masses follow the same distribution.
This is the basis of the spectral siren method of measuring cosmological parameters; we can simultaneously infer the relationship between redshift and luminosity distance with the parameters of the source frame mass distribution.
The full set of hyper-parameters therefore includes the cosmological parameters that dictate the $D_L$-$z$ relation: the local expansion rate \Ho, the present fractional energy density of dark matter \Omm, dark energy $\Omega_\Lambda$ and radiation $\Omega_r$, and the equation of state of dark energy $w$.
In this work, we fix $\Omega_\Lambda=1-\Omega_M, \Omega_r=0$ and $w=-1$ and use uniform priors on \Ho{} and \Omm{}. 
This corresponds to a flat $\Lambda$CDM cosmology.

Algorithmically, we evaluate the likelihood in Equation~\ref{eq:inhomog-poisson} by drawing \Ho{} and \Omm{} from their prior distributions (contained in $p(\Lambda)$), using them to define a relationship between luminosity distance and redshift, then transforming detector frame masses to source frame masses according to $m_{\det} /(1+z(D_L,H_0,\Omega_M)) = m_{\source}$, evaluating the source-frame mass distribution at these transformed values ($\frac{\diff N}{\diff t_{\det} \diff \theta_i} (\theta_i;\Lambda)$ for events and $N_{\text{exp}}(\Lambda)$ for injections), and then taking the product of these terms to evaluate the expression in Equation~\ref{eq:inhomog-poisson}.
The full process is outlined in Section~\ref{sec:model}.

\subsection{\Acl{GP}-based mass distribution}
\label{sec:model}

Our goal is to model the population of \ac{GW} in an astrophysically-agnostic way. 
There exist several non-parametric methods developed for this purpose \citep{tiwari_vamana_2021,edelman_aint_2022,sadiq_flexible_2022,edelman_cover_2023,mandel_extracting_2019,ray_non-parametric_2023}.
While well-suited to infer the \ac{GW} source population with a fixed cosmology, several of these methods employ fixed features in source frame mass, such as bin edges \citep{mandel_extracting_2019,ray_non-parametric_2023} or spline nodes \citep{edelman_aint_2022}.
Since these locations were chosen with a fixed cosmology, they risk causing the inference to prefer the cosmological parameters used when choosing the feature locations. 
Indeed, \citet{mastrogiovanni_importance_2021} show how using fixed features can significantly bias cosmological inference within the spectral siren methodology.
We therefore opt for a model of the source frame mass distribution that foregoes the need to define such features.

For this purpose, we construct a model with a \acf{GP}, a common tool for non-parametric inference. 
\Acp{GP} are random processes for which any linear combination of outcomes are Gaussian distributed \citep{rasmussen_gaussian_2006}.
Their smoothness properties make them widely useful in \ac{GW} data analysis and beyond for regression problems, such as modeling time-domain waveforms \citep{doctor_statistical_2017, huerta_eccentric_2018} and the neutron star equation of state \citep{landry_nonparametric_2019}, density estimation problems, such as estimating posterior densities of single-event parameters from parameter estimation samples ~\citep{demilio_density_2021}, and as a prior on histogram bin heights for population inference \citep{mandel_model-independent_2017, li_flexible_2021, ray_non-parametric_2023}.

Our use case is slightly different from previous analyses:
we utilize a \ac{GP} as a prior on the functions that describe the primary mass distribution of \acp{CBC}.
It will therefore replace $p(\Lambda)$ in Equation~\ref{eq:inhomog-poisson}, except in the case of cosmological parameters, which all have uniform priors.
This choice adds very little prior information about the shape of the mass distribution, besides enforcing that it must be smooth.
Practically, the only difference in the inference of the population when using a GP versus other modeling choices is that $\frac{\diff N}{\diff t_{\det} \diff \theta_i} (\theta_i;\Lambda)$ in Equation~\ref{eq:inhomog-poisson} is determined directly by a realization of the \ac{GP}, rather than by a handful of hyper-parameters $\Lambda$ and evaluated on an analytical function.
In other words, when using parametric models, $\frac{\diff N}{\diff t_{\det} \diff \theta_i} (\theta_i;\Lambda)$ in Equation~\ref{eq:inhomog-poisson} is calculated by evaluating a specific functional form described by a small set of hyper-parameters. 
With the \ac{GP} approach, the hyper-parameters describing the mass distribution are the rate at each event-level posterior sample's source frame mass, and the rate at each found injection's source frame mass.
% The source frame mass is different for each possible value of \Ho{} and \Omm{}, so the hyper-parameters are defined in different locations for each draw from the hyper-posterior.
In this way, our \ac{GP}-based mass distribution is similar to the autoregressive population models used in ~\citet{callister_parameter-free_2023}.
Indeed, an autoregressive process is a \ac{GP} with a particular choice of kernel (described below).
We therefore have $N_{\text{evs}}M + N_{\text{inj}}$ mass hyper-parameters, where $N_{\text{ev}}$ is the number of events, $M$ is the number of posterior samples per event, and $N_{\text{inj}}$ is the number of injections used to calculate the selection function $\xi$.
We label these $\{\mathcal{R}_{ij},\mathcal{R}_k\}$, and our full set of hyper-parameters is $\Lambda=\{H_0, \Omega_M, \mathcal{R}_{ij},\mathcal{R}_k\}$. 
As the hyper-parameters differ at each data point, we have successfully fulfilled our goal to construct a model without pre-defined features in source frame mass.
We note that other non-parametric methods may be adapted to avoid such pre-defined features, such as fitting for the locations of their features simultaneously the rest of the inference \citep[e.g.][]{tiwari_vamana_2021} or by using an appropriate smoothing \citep[e.g.][]{edelman_cover_2023}. 
Such adaptations are necessary to avoid circular inference.

The smoothness of a given \acp{GP} is determined by its kernel, which is a function that defines the covariance between input points in the \ac{GP} (in our case, two source frame mass values). 
It defines the notion of similarity between adjacent points and thereby encodes our assumptions about the smoothness of the source frame mass distribution \citep{rasmussen_gaussian_2006}.
We use a Mat\'ern kernel \citep{handcock_bayesian_1993, stein_interpolation_1999} with $\nu = 5/2$, but have repeated the analysis with $\nu=3/2$ and $\infty$, finding little impact on the results, except that the $\nu=\infty$ case (also called the ``squared exponential'' kernel) produces a slightly more jagged mass distribution.
Mat\'ern kernels have two parameters besides the mean that determine their properties: a length scale $l$ and variance $s$.
In our use case, these are one level further removed from hyper-parameters, so we adopt the terminology used in \citet{callister_parameter-free_2023} and call them ``hyper-hyper-parameters.''
We fit these hyper-hyper-parameters along with the hyper-parameters $\Lambda$ to minimize prior assumptions about the form of the mass distribution.
We use ``penalized-complexity'' priors on the hyper-hyper-parameters to enforce that the model does not create small-scale structure uninformed by data, thereby avoiding over-fitting \citep{simpson_penalising_2017,simpson_garcpas_2022}. 
Explicitly, the priors on $l$ and $s$ are Fr\'echet and Gamma distributions, respectively, and are defined to have less than 5\% support for correlation lengths smaller than the average spacing between event-level posterior means.

The time to evaluate a \acp{GP} notoriously scales as the cube of the number of data points, making \acp{GP} unwieldy with large data sets such as the ones expected for \ac{O5}.
We therefore make two approximations to a full \ac{GP} to increase computational efficiency.
First, for each likelihood evaluation, we evaluate a full \ac{GP} on a regular grid between $0.1\Msun$ and $250\Msun$ and then interpolate it at each data point.
Second, we use the quasi-separability of Mat\'ern kernels to analytically perform the transformation between covariance matrix and \ac{GP} draw.
This second step is done using the \texttt{QuasisepSolver} module \citep{foreman-mackey_fast_2017} in the \texttt{tinygp} code base \citep{foreman-mackey_tinygp_2021}, and requires data to be sortable (i.e. one-dimensional).

Algorithmically, each likelihood evaluation contains the following steps: %\variable{output/priors_placeholder.txt}
\begin{enumerate}
\script{nonparametric_inference.py}
    \item Draw cosmological parameters \Ho{} and \Omm{} from uniform prior distributions.
    \item Transform the luminosity distances and detector frame masses of each event posterior sample and injection into redshifts and source frame masses according to the cosmology specified by step 1.
    \item Draw hyper-hyper-parameters $l$ and $s$ from the penalized-complexity priors described above.
    \item Draw a single \ac{GP} realization with a kernel defined by $l$ and $s$.
    This is defined on a regular grid of source-frame masses and evaluated using the \texttt{QuasisepSolver} in \texttt{tinygp}.
    \item Interpolate the \ac{GP} at each event posterior sample and injection source frame mass (from step 3)
    \item Calculate the population likelihood according to Equation~\ref{eq:inhomog-poisson}.
\end{enumerate}
\jme{[comment: is there a way to summarize this list in a simple diagram?]}


\section{Results}
\label{sec:results}
In this Section, we first show that fitting an incorrect functional form to the mass distribution of \acp{CBC} biases the inference of cosmological parameters when using the spectral siren methodology.
We then demonstrate that our flexible model alleviates this bias without the need to know the mass distribution's morphology \emph{a priori}.
To illustrate this, we generate a catalog of $\variable{output/num_found_events.txt}$ \ac{GW} events from an underlying mass distribution described by the \plp{} model, with hyperparameters consistent with those found in \citet{abbott_population_2023}.
This simulated catalog is designed to be representative of the \ac{GW}  transient catalog after one year of observation in \ac{O5}.
The details of the data simulation are described in Appendix~\ref{ap:data generation}.
We then use this catalog to infer the cosmic expansion rate, using three different models for the source-frame mass distribution:
\begin{enumerate}
    \item the \plp{} model, which includes the true mass distribution within its hyperprior,
    \item a broken power law \jme{[use same font style as \plp{}?]}, which does not include the true mass distribution within its hyperprior, and 
    \item the flexible, \ac{GP}-based model described in Section~\ref{sec:model}, which is able to closely approximate the morphology of the true mass distribution, along with many other morphologies.
\end{enumerate}
The results of each of these fits are shown in Figure~\ref{fig:O5_GP}. 
The left panel shows the inferred source frame mass distribution for each of the considered models, and the right panel shows the corresponding posteriors on \Ho{}. 
We indicate the true underlying source mass distribution and fiducial \Ho{} value with solid black lines in each panel. 

We find that using a parametric model of the same form as the true simulated distribution (\plp) recovers the injected value of \Ho{} within $\variable{output/PLPh0offset.txt}\sigma$.
However, when using a model that cannot accurately represent the true shape of the mass distribution, we recover a biased estimate of \Ho: the true value is offset from the mean of the posterior by $\variable{output/BPLh0offset.txt}\sigma$.
We have repeated this analysis with 50 separate simulated catalogs of $\sim500$ events, and found that the broken power law model produces an overestimate of \Ho{} at greater than $1\sigma$ %$\variable{output/BPL_bias_percent.txt}\%$ 
of the time of the time, whereas %the \ac{GP}-based model reaches that level of bias only \result{Z}\% of the time, and 
the \plp{} model reaches the same level of bias %$\variable{output/PLP_bias_percent.txt}\%$ 
of the time.

This bias is not due to the need to know the morphology of the mass distribution \emph{a priori}, as the \ac{GP}-based model recovers the correct value of \Ho{} despite making minimal assumptions about the mass distribution: the true value is recovered at the $\variable{output/nonparh0offset.txt}\sigma$ level.
The \ac{GP}-based model does, however, obtain larger uncertainties on the inferred value of \Ho, as it has many more free parameters. 
In reality, it is impossible to know the true functional form of the mass distribution, so the additional statistical uncertainty introduced by non-parametric approaches may be desirable over the systematic errors introduced by choosing a parametric model.

Additionally, the mass distribution inferred with the \ac{GP}-based model closely resembles the true, simulated distribution.
At very high and very low masses, there is both a lack of data and little detector sensitivity.
The mass distribution is therefore uninformed in this region and the \ac{GP} reverts to its prior distribution, which resembles random scatter around the mean differential merger rate.
Similar effects can be seen in other non-parametric methods \citep{edelman_cover_2023, callister_parameter-free_2023}.
On the other hand, the lack of data just below the minimum black hole mass and just above the maximum black hole mass, combined with the fact that \ac{GW} detectors are sensitive to objects in that mass range causes the \ac{GP} to learn a relatively low merger rate at the edges of the mass distribution.
Indeed, these features are very informative to the \Ho{} constraint, especially at high masses for current detectors \citep{gwtc3_cosmo}.
The combination of these two effects results in what appears as an uptick in the merger rate at very low and very high masses, but these features are uninformative and do not affect inference on cosmological parameters.
Additionally, the posterior on \Ho{} is distinct from its prior distribution (uniform in the range $[30\Hunits,120\Hunits]$, indicating that the data is informative despite the flexibility of the population model.

Collectively, these results indicate that prior knowledge of the shape of the mass distribution is not required to perform a spectral siren measurement, so long as strong assumptions about the shape of the mass distribution are not made.

\subsection{Projections for Future Measurements}
The fit shown in Figure~\ref{fig:O5_GP} implies that we can expect to measure \Ho{} to $\variable{output/nonparh0percent.txt}\%$ ($1-\sigma$ uncertainties) after one year of \ac{O5} using the \ac{GP}-based spectral siren method.
However, by the time of \ac{O5}, the \ac{GW} detector network is projected to detect \acp{BBH} up to redshift $\sim 3$, with most sources lying near redshift $\sim 1.2$ \citep{chen_distance_2021}.
Additionally, next-generation detectors will be sensitive to sources up to redshift $\sim 40$.
This means that future \acp{GW} observations will be more sensitive $H(z\gtrsim1)$ than to \Ho, and can therefore constrain several additional cosmological parameters \citep{Chen:2024gdn}.
We demonstrate this by repeating the same \ac{GP} spectral siren analysis of the previous section while also fitting for the local matter density, \Omm, but we emphasize that it would also have been possible to constrain other cosmological parameters governing $H(z)$.
The result is shown in Figure~\ref{fig:O5_corner}.

We find \ac{O5} observations to be most sensitive to $H(z=\variable{output/mostsensitivez.txt})$, which results in similarly informative constraints on \Omm{} and \Ho.
The left panel of Figure~\ref{fig:O5_corner} demonstrates a strong anti-correlation between the \Omm{} and \Ho{} posteriors. 
This is because \Ho{} controls the $y$-intercept of the $H(z)$ curves on the right panel, while \Omm{} informs the slope of those curves.
The same measurement of $H(z\neq0)$ can be obtained by increasing the slope while decreasing the $y$-intercept, and vice versa.
Similar behavior can be observed in measurements of the \ac{BBH} redshift distribution, which exhibit a ``waist'' at $z\sim0.2$ with current observations \citep{abbott_population_2023, callister_parameter-free_2023}.

Next-generation detectors will be sensitive to a much larger range of redshifts \citep{et_steering_committee_einstein_2020, evans_horizon_2021} and will therefore break the degeneracy between cosmological parameters, allowing for tighter constraints on each.
However, the low cosmological volume (and thus low number of mergers) at low redshift will always limit the constraining power of spectral sirens at $z=0$, potentially making this method more sensitive to cosmological parameters that affect higher redshifts than to \Ho.
Combining spectral sirens with other methods that are more sensitive to the local expansion rate, such as those that employ electromagnetic counterparts, can increase the precision of \ac{GW} standard sirens at all redshifts \citep[e.g.][]{Chen:2024gdn}.

\begin{figure*}
    \centering
    \includegraphics[width=\textwidth]{figures/O5_GP_corner.pdf}
    \caption{
    Projected constraints on multiple cosmological parameters after one year of observing at the \ac{LVK}'s design sensitivity, using the \acl{GP}-based spectral siren method.
    The right panel shows the inferred expansion history of the universe, $H(z)$.
    It will be measured most precisely at $z=\variable{output/mostsensitivez.txt}$, as can be seen by the narrowing of the inferred $H(z)$ curves there.
    The inset shows the posterior on $H(z=\variable{output/mostsensitivez.txt})$.
    Black solid lines indicate the true value of $H(z)$ in both the inset and main panel.
    The left panel shows the two-dimensional posterior on \Ho{} and \Omm, with the true value indicated by a black ``+''.
    The two parameters are strongly degenerate because of the multiple ways of measuring $z=\variable{output/mostsensitivez.txt}$.
    Spectral sirens are particularly well-suited to measuring cosmological parameters that affect the mid-to-high redshift universe. 
    }
    \label{fig:O5_corner}
    \script{nonparametric_corner.py}
\end{figure*}

% \subsubsection{BONUS: XG}
% To demonstrate the capabilities of next-generation detectors, we repeat the full analysis with \result{1000} events indicative of Cosmic Explorer's projected best-measured events within the first month alone.
% We fit the dark energy equation of state parameter, $w$, along with \Ho, and show results in Figure~\ref{fig:XG_corner}.
% We recover unbiased measurements of each parameter and find a \result{X}\% measurement of $w$ and a \result{X}\% measurement of \Ho.
% In comparison, the most recent constraints on these parameters with \result{X experiment} were \result{X}\% and \result{X}\% , respectively.
% This illustrates the unique ability of \acp{GW} to probe the high-redshift universe.
% 
% \begin{figure*}
%     \centering
%     \includegraphics[width=\textwidth]{figures/CE_GP.pdf}
%     \caption{Two-panel figure of (A) $w$, and \Ho{} corner plot and (B)$H(z)$ for CE. We can include a posterior of $H(z=z_{\text{best measured}})$ as an inset to the $H(z)$ plot.}
%     \label{fig:XG_corner}
%     \script{XG_corner.py}
% \end{figure*}

\section{Discussion}
\label{sec:discussion}

\acp{GW} are unique cosmic messengers in that they carry both redshift and distance information, making them remarkably clean probes of the Universe's expansion history. 
However, the current method of determining \ac{GW} redshifts via the distribution of their source masses (i.e. spectral sirens) employs an assumption on the shape of their population, typically enforced with simplified parametric functions. 
This choice of mass distribution, often presumed to be necessary or fundamental to the method, may introduce a systematic bias to a measurement of the cosmological parameters that otherwise makes very few assumptions.
    
In this work, we show that such a choice is not in fact necessary.
We do this by accurately measuring cosmological parameters with a highly flexible model for the mass distribution. 
The success of the flexible model in recovering an informative and accurate posteriors on \Ho{} and \Omm{} demonstrates that there is no need to know the \ac{CBC} mass distribution's morphology in order to constrain cosmological parameters with it.
This reinforces the notion that the information in the spectral siren measurement comes from the assumption that all \ac{GW} sources come from the same population, a far less stringent statement than the assumption that we understand the exact astrophysical processes which give rise to that population (i.e. the physics governing compact binary formation and evolution).
    
We have also predicted a $\variable{output/nonparh0percent.txt}\%$ measurement of \Ho{} after one year of observing with the \ac{LVK} network's design sensitivity using non-parametric spectral sirens and a $\variable{output/PLPh0percent.txt}\%$ 
measurement with parametric spectral sirens, demonstrating an increase in precision when using the correct parametric model.
However, the true population of merging compact binaries in the Universe is unknown because neither the mechanism from which compact binaries were formed, nor the physical processes within that mechanism has been conclusively determined.
It is therefore impossible to know the true functional form of the mass distribution, so systematic uncertainties arising from an incorrect choice for the form of the mass distribution are inevitable.
With current observations, these systematic effects are dwarfed by statistical uncertainties.
However, next-generation detectors will herald in enough \ac{GW} observations to substantially decrease statistical uncertainty in these measurements: \citet{pierra_study_2023} show that incorrect assumptions about the shape of the mass distribution can lead to $\sim3\sigma$ systematic biases in \Ho{} when catalogs contain 2,000 events, though this bias may be an over-estimate as it does not include measurement uncertainty of \ac{GW} parameters.
Thus, the additional statistical uncertainty introduced by non-parametric approaches may be preferable to the systematic errors associated with choosing a parametric model.

Work in preparation will extend the method developed here to mass distributions that are allowed to evolve with redshift.
Non-parametric methods will allow for arbitrary redshift evolution, but must constrain that evolution to avoid exactly mimicking the effects of cosmological redshift.
As demonstrated in \citet{ezquiaga_spectral_2022}, the degeneracy between an evolving mass distribution and the expansion of the universe can be broken provided there are multiple features present in the mass distribution and all features do not identically and monotonically move to higher source frame mass with increasing redshift.
The first condition (multiple features) is known to be met in current data \citep{abbott_population_2023}, with three robust features \citep{farah_things_2023}: a maximum black hole mass and overdensities at $\sim10\Msun$ and $\sim35\Msun$.
The latter scenario (features identically and monotonically moving to higher masses with increasing redshift) would be astrophysically unlikely as the the locations of features in the mass distribution are each thought to be governed by different physical processes \citep{mapelli_binary_2020}.
Since \aclp{GP} naturally scale to multiple data dimensions, \citep{rasmussen_gaussian_2006}, the method presented here can be straightforwardly generalized to fitting the redshift dependence of the mass distribution. 

Because \ac{GW} observations are the only data input to spectral sirens, they are uniquely positioned to probe the expansion history of the universe over a wide range of cosmological redshift, rather than just the local expansion \Ho.
We have shown this by simultaneously measuring \Omm{} and \Ho{}, but the method can be trivially extended to constrain other cosmological parameters that govern $H(z)$, such as the dark energy equation of state parameter, $w$. 
We find that spectral sirens will be most sensitive to $z=\variable{output/mostsensitivez.txt}$ during \acf{O5}.
This redshift is larger than the expected redshifts of detectable electromagnetic counterparts of binary neutron star mergers ~\citep{kiendrebeogo_updated_2023}\jme{, which can only be detected relatively locally}.
meaning \jme{that} upgrades to current \ac{GW} detectors will allow the spectral siren method to probe $H(z)$ at otherwise unexplored distances \jme{by \ac{GW}}. 
Proposed next generation \jme{\ac{GW}} detectors such as Cosmic Explorer and Einstein Telescope will be sensitive to \jme{coalescence binaries across cosmic time}$z\sim40$.
At that time, cosmological surveys such as those enabled by the Nancy Grace Roman Space Telescope and the Vera Rubin Observatory are expected to be able to precisely measure $H(z)$ to $z\sim3$ \citep{spergel_wide-field_2015}, making the spectral siren method uniquely positioned to measure the expansion of the universe at high redshift.
When combined with other standard siren methods, the low-redshift expansion will also be well-constrained \citep{Chen:2024gdn}.
While the sensitivity to high redshifts is a feature of the spectral siren method in general, non-parametric methods such as the one presented here will be imperative to avoid systematic biases as next-generation detectors provide additional data. 

\begin{acknowledgments}
    The authors thank Reed Essick, Utkarsh Mali, Ben Farr, Maya Fishbach, Colm Talbot,  for helpful conversations. 
    %Amanda
    A.M.F. is supported by the National Science Foundation Graduate Research Fellowship Program under Grant No. DGE-1746045.
    %Tom
    %Jose
    J.M.E. is supported by the European Union’s Horizon 2020 research and innovation program under the Marie Sklodowska-Curie grant agreement No. 847523 INTERACTIONS, and by VILLUM FONDEN (grant no. 53101 and 37766). 
    %Mike
    M.Z. gratefully acknowledges funding from the Brinson Foundation in support of astrophysics research at the Adler Planetarium.
    %Daniel
    D.E.H is supported by NSF grants AST-2006645 and PHY-2110507, as well as by the Kavli Institute for Cosmological Physics through an endowment from the Kavli Foundation and its founder Fred Kavli.
    %Cluster
    The Tycho supercomputer hosted at the SCIENCE HPC center at the University of Copenhagen was used for supporting this work.
\end{acknowledgments}

\software{\texttt{numpyro} \citep{phan_composable_2019,bingham_pyro_2019}, \texttt{tinygp} \citep{foreman-mackey_tinygp_2021}, \texttt{arviz} \citep{kumar_arviz_2019}, \texttt{jax}, \showyourwork \citep{Luger2021}
          }

\bibliography{bib}
\appendix
\section{Details of data simulation}
\label{ap:data generation}

We generate a mock catalog of \ac{GW} events characteristic of that expected from one year of \acf{O5}. 
These events are drawn from an underlying population described by the \plp{} mass distribution presented in \citet{talbot_measuring_2018} and used in \citet{abbott_binary_2019, abbott_population_2023} and the redshift distribution presented in \citet{callister_shouts_2020}.
\begin{equation}
\label{eq:underlying population}
    \frac{\diff N}{\diff m_1 \diff z} \propto p(m_1|\bar{\Lambda}_m) p(z|\bar{\Lambda}_z, H_0, \Omega_M),
\end{equation}
where 
\begin{align}
    p(m_1,m_2|\bar{\Lambda}_m) &\propto \mathcal{S}(m_{\min},m_{\max})
    \left( f_{\text{peak}}e^{-\frac{1}{2}(\frac{m_1-\mu}{\sigma})^2}\mathcal{N}_{\text{g}} +
    (1-f_{\text{peak}})m_1^{\alpha}\mathcal{N}_{\text{pl}} \right) ,
\label{eq:underlying mass dist}
\end{align}
\begin{equation}
    p(z|H_0, \Omega_M) \propto \frac{\diff V_C}{\diff z} \frac{1}{1+z} \frac{(1+z)^\alpha}{1+\left(\frac{1+z}{1+z_p}\right)^{\alpha+\beta}}.
    \label{eq:underlying redshift dist}
\end{equation}
Here, $V_C(H_0, \Omega_M)$ is the comoving volume for given cosmological parameters \Ho{} and \Omm{}, and $\bar{\Lambda}_m = \{\alpha, m_{\min}, m_{\max}, \mu, \sigma, f_{\text{peak}}\}$ are the (hyper-)parameters describing the power law in primary mass, minimum and maximum black hole mass, Gaussian peak location and width, and fraction of events in the Gaussian peak, respectively.
$\mathcal{S}$ is a smoothing function at low and high masses, and $\mathcal{N}_{\text{pl}}$ and $\mathcal{N}_{\text{g}}$ are the normalizations between  $m_{\min}$ and $m_{\max}$ for the power law component and truncated Gaussian component, respectively.
$\bar{\Lambda}_z = \{z_p,\alpha_z,\beta_z\}$ are the parameters governing the redshift distribution's peak redshift, low-$z$ power law slope, and high-$z$ power law slope.
When generating the simulated events, we have fixed $\alpha=-2.7,m_{\max}=78\Msun,m_{\min}=10\Msun, \mu=30\Msun,\sigma=7.0\Msun, f_{\text{peak}}=0.05,   z_p=2.4,\alpha_z=1.,\beta_z=3.4$.
These choices correspond to the maximum \emph{a posteriori} values obtained by an analysis of GWTC-3 data using the \plp{} model \citep{abbott_population_2023}.
We do not fit for secondary masses or spins, as they do not carry additional cosmological information.
We use cosmological parameters \Ho$=\variable{output/H0_FID.txt}\Hunits$, \Omm$=0.3$, $\Omega_\Lambda=1-$\Omm, consistent with those found by \citet{planck_collaboration_planck_2016}.
We emphasize, however, that the choice of cosmological parameters for data generation is arbitrary and does not impact the results, since we are concerned only with the ability of our method to recover the injected values.
Throughout the data generation and inference, we use the approximations presented in \citet{adachi_analytical_2012} to efficiently convert between $d_L$ and $z$ for a given set of cosmological parameters \Omm{} and \Ho.

We do not include neutron stars in our simulation set, as their contribution to the spectral siren measurement is expected to be subdominant in \ac{O5}.
However, if a lower mass gap between the heaviest neutron stars and lightest black holes exists, it will provide an additional feature with which to inform the measurement, and will be the most informative feature for spectral siren measurements with next generation detectors \citep{ezquiaga_spectral_2022}.

After passing the simulated events through projected detector selection effects, the resulting catalog has $\variable{output/num_found_events.txt}$ events, consistent with the numbers projected for \ac{O5} by \citet{kiendrebeogo_updated_2023}.
We use the software package \texttt{GWMockCat} \citep{farah_things_2023} to simulate posterior samples for these events with measurement uncertainties typical of those expected from \ac{O5} detectors.
\texttt{GWMockCat} also simulates a set of software injections, which we use to estimate selection effects in the inference.
To determine the detectability of both injections and simulated events in O5, we use the projected \ac{O5} LIGO power spectral density \citep{obsscen_noise_curves,abbott_prospects_2020} to calculate observed signal-to-noise ratios $\rho_{\text{obs}}$, and we consider events and injections with $\rho_{\text{obs}}>8$ to be detectable. 
The full procedure for this mock data generation process is described in \citet{fishbach_where_2017, farah_things_2023, essick_ensuring_2024}.

\section{Old}
Old tex that we may end up needing.

\begin{table}[]
    \centering
    \begin{tabular}{c|c c}
         Model & $\sigma_{H_0}$ [km/s/Mpc] & bias [km/s/Mpc]\\
         \hline
         & 
    \end{tabular}
    \caption{Performance of various mass distribution models used for measuring the Hubble constant, \Ho.
    We consider the correct parametric model, a deliberately incorrect parametric model, and the non-parametric, \ac{GP}-based model presented in this work.
    Bias is defined as the number of standard deviations between the true injected value of \Ho and the recovered posterior mean. 
    The parametric models yield more precise constraints on \Ho{}, as shown by smaller standard deviations ($\sigma_{H_0}$), but the incorrect parametric model is less accurate than the non-parametric model.
    In reality, it is impossible to know the true functional form of the mass distribution, so the additional statistical uncertainty introduced by non-parametric approaches may be desirable over the systematic errors associated with choosing a parametric model.
    }
    \label{tab:bias}
\end{table}

\subsection{Option 2: GP only}
We fit the non-parametric population model described in Section~\ref{sec:model} to a simulated \ac{GW} catalog of $\variable{output/num_found_events.txt}$ \ac{GW} events from an underlying mass distribution described by the \plp{} model, with hyperparameters consistent with those found in \citet{abbott_population_2023}.
This simulated catalog is designed to be representative of the \ac{GW}  transient catalog after \ac{O5}.
The details of the data simulation are described in Appendix~\ref{ap:data generation}.

The resulting inferred mass distribution is shown in Figure~\ref{fig:O5_GP}, along with the corresponding posterior on \Ho.
The inferred mass distribution closely resembles the true, simulated distribution, and the injected value of \Ho{} ($\variable{output/H0_FID.txt}\Hunits$) is consistent with its inferred posterior (\variable{output/nonparh0CI.txt}$\Hunits$).
At very high and very low masses, there is both a lack of data and little detector sensitivity.
The mass distribution is therefore uninformed in this region and the \ac{GP} reverts to its prior distribution, which resembles random scatter around the mean differential merger rate.
Similar effects can be seen in other non-parametric methods \citep{edelman_cover_2023, callister_parameter-free_2023}.
On the other hand, the lack of data just below the minimum black hole mass and just above the maximum black hole mass, combined with the fact that \ac{GW} detectors are sensitive to objects in that mass range causes the GP to learn a relatively low merger rate at the edges of the mass distribution.
Indeed, these features are very informative to the \Ho{} constraint \citep{gwtc3_cosmo}.
The combination of these two effects results in what appears as an uptick in the merger rate at very low and very high masses, but these features are uninformative and do not affect inference on cosmological parameters.
Additionally, the posterior on \Ho{} is distinct from its prior distribution (uniform in the range $[30\Hunits,120\Hunits]$, indicating that the data is informative despite the flexibility of the population model.

\end{document}
